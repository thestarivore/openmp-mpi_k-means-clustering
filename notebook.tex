
% Default to the notebook output style

    


% Inherit from the specified cell style.




    
\documentclass[11pt]{article}

    
    
    \usepackage[T1]{fontenc}
    % Nicer default font (+ math font) than Computer Modern for most use cases
    \usepackage{mathpazo}

    % Basic figure setup, for now with no caption control since it's done
    % automatically by Pandoc (which extracts ![](path) syntax from Markdown).
    \usepackage{graphicx}
    % We will generate all images so they have a width \maxwidth. This means
    % that they will get their normal width if they fit onto the page, but
    % are scaled down if they would overflow the margins.
    \makeatletter
    \def\maxwidth{\ifdim\Gin@nat@width>\linewidth\linewidth
    \else\Gin@nat@width\fi}
    \makeatother
    \let\Oldincludegraphics\includegraphics
    % Set max figure width to be 80% of text width, for now hardcoded.
    \renewcommand{\includegraphics}[1]{\Oldincludegraphics[width=.8\maxwidth]{#1}}
    % Ensure that by default, figures have no caption (until we provide a
    % proper Figure object with a Caption API and a way to capture that
    % in the conversion process - todo).
    \usepackage{caption}
    \DeclareCaptionLabelFormat{nolabel}{}
    \captionsetup{labelformat=nolabel}

    \usepackage{adjustbox} % Used to constrain images to a maximum size 
    \usepackage{xcolor} % Allow colors to be defined
    \usepackage{enumerate} % Needed for markdown enumerations to work
    \usepackage{geometry} % Used to adjust the document margins
    \usepackage{amsmath} % Equations
    \usepackage{amssymb} % Equations
    \usepackage{textcomp} % defines textquotesingle
    % Hack from http://tex.stackexchange.com/a/47451/13684:
    \AtBeginDocument{%
        \def\PYZsq{\textquotesingle}% Upright quotes in Pygmentized code
    }
    \usepackage{upquote} % Upright quotes for verbatim code
    \usepackage{eurosym} % defines \euro
    \usepackage[mathletters]{ucs} % Extended unicode (utf-8) support
    \usepackage[utf8x]{inputenc} % Allow utf-8 characters in the tex document
    \usepackage{fancyvrb} % verbatim replacement that allows latex
    \usepackage{grffile} % extends the file name processing of package graphics 
                         % to support a larger range 
    % The hyperref package gives us a pdf with properly built
    % internal navigation ('pdf bookmarks' for the table of contents,
    % internal cross-reference links, web links for URLs, etc.)
    \usepackage{hyperref}
    \usepackage{longtable} % longtable support required by pandoc >1.10
    \usepackage{booktabs}  % table support for pandoc > 1.12.2
    \usepackage[inline]{enumitem} % IRkernel/repr support (it uses the enumerate* environment)
    \usepackage[normalem]{ulem} % ulem is needed to support strikethroughs (\sout)
                                % normalem makes italics be italics, not underlines
    

    
    
    % Colors for the hyperref package
    \definecolor{urlcolor}{rgb}{0,.145,.698}
    \definecolor{linkcolor}{rgb}{.71,0.21,0.01}
    \definecolor{citecolor}{rgb}{.12,.54,.11}

    % ANSI colors
    \definecolor{ansi-black}{HTML}{3E424D}
    \definecolor{ansi-black-intense}{HTML}{282C36}
    \definecolor{ansi-red}{HTML}{E75C58}
    \definecolor{ansi-red-intense}{HTML}{B22B31}
    \definecolor{ansi-green}{HTML}{00A250}
    \definecolor{ansi-green-intense}{HTML}{007427}
    \definecolor{ansi-yellow}{HTML}{DDB62B}
    \definecolor{ansi-yellow-intense}{HTML}{B27D12}
    \definecolor{ansi-blue}{HTML}{208FFB}
    \definecolor{ansi-blue-intense}{HTML}{0065CA}
    \definecolor{ansi-magenta}{HTML}{D160C4}
    \definecolor{ansi-magenta-intense}{HTML}{A03196}
    \definecolor{ansi-cyan}{HTML}{60C6C8}
    \definecolor{ansi-cyan-intense}{HTML}{258F8F}
    \definecolor{ansi-white}{HTML}{C5C1B4}
    \definecolor{ansi-white-intense}{HTML}{A1A6B2}

    % commands and environments needed by pandoc snippets
    % extracted from the output of `pandoc -s`
    \providecommand{\tightlist}{%
      \setlength{\itemsep}{0pt}\setlength{\parskip}{0pt}}
    \DefineVerbatimEnvironment{Highlighting}{Verbatim}{commandchars=\\\{\}}
    % Add ',fontsize=\small' for more characters per line
    \newenvironment{Shaded}{}{}
    \newcommand{\KeywordTok}[1]{\textcolor[rgb]{0.00,0.44,0.13}{\textbf{{#1}}}}
    \newcommand{\DataTypeTok}[1]{\textcolor[rgb]{0.56,0.13,0.00}{{#1}}}
    \newcommand{\DecValTok}[1]{\textcolor[rgb]{0.25,0.63,0.44}{{#1}}}
    \newcommand{\BaseNTok}[1]{\textcolor[rgb]{0.25,0.63,0.44}{{#1}}}
    \newcommand{\FloatTok}[1]{\textcolor[rgb]{0.25,0.63,0.44}{{#1}}}
    \newcommand{\CharTok}[1]{\textcolor[rgb]{0.25,0.44,0.63}{{#1}}}
    \newcommand{\StringTok}[1]{\textcolor[rgb]{0.25,0.44,0.63}{{#1}}}
    \newcommand{\CommentTok}[1]{\textcolor[rgb]{0.38,0.63,0.69}{\textit{{#1}}}}
    \newcommand{\OtherTok}[1]{\textcolor[rgb]{0.00,0.44,0.13}{{#1}}}
    \newcommand{\AlertTok}[1]{\textcolor[rgb]{1.00,0.00,0.00}{\textbf{{#1}}}}
    \newcommand{\FunctionTok}[1]{\textcolor[rgb]{0.02,0.16,0.49}{{#1}}}
    \newcommand{\RegionMarkerTok}[1]{{#1}}
    \newcommand{\ErrorTok}[1]{\textcolor[rgb]{1.00,0.00,0.00}{\textbf{{#1}}}}
    \newcommand{\NormalTok}[1]{{#1}}
    
    % Additional commands for more recent versions of Pandoc
    \newcommand{\ConstantTok}[1]{\textcolor[rgb]{0.53,0.00,0.00}{{#1}}}
    \newcommand{\SpecialCharTok}[1]{\textcolor[rgb]{0.25,0.44,0.63}{{#1}}}
    \newcommand{\VerbatimStringTok}[1]{\textcolor[rgb]{0.25,0.44,0.63}{{#1}}}
    \newcommand{\SpecialStringTok}[1]{\textcolor[rgb]{0.73,0.40,0.53}{{#1}}}
    \newcommand{\ImportTok}[1]{{#1}}
    \newcommand{\DocumentationTok}[1]{\textcolor[rgb]{0.73,0.13,0.13}{\textit{{#1}}}}
    \newcommand{\AnnotationTok}[1]{\textcolor[rgb]{0.38,0.63,0.69}{\textbf{\textit{{#1}}}}}
    \newcommand{\CommentVarTok}[1]{\textcolor[rgb]{0.38,0.63,0.69}{\textbf{\textit{{#1}}}}}
    \newcommand{\VariableTok}[1]{\textcolor[rgb]{0.10,0.09,0.49}{{#1}}}
    \newcommand{\ControlFlowTok}[1]{\textcolor[rgb]{0.00,0.44,0.13}{\textbf{{#1}}}}
    \newcommand{\OperatorTok}[1]{\textcolor[rgb]{0.40,0.40,0.40}{{#1}}}
    \newcommand{\BuiltInTok}[1]{{#1}}
    \newcommand{\ExtensionTok}[1]{{#1}}
    \newcommand{\PreprocessorTok}[1]{\textcolor[rgb]{0.74,0.48,0.00}{{#1}}}
    \newcommand{\AttributeTok}[1]{\textcolor[rgb]{0.49,0.56,0.16}{{#1}}}
    \newcommand{\InformationTok}[1]{\textcolor[rgb]{0.38,0.63,0.69}{\textbf{\textit{{#1}}}}}
    \newcommand{\WarningTok}[1]{\textcolor[rgb]{0.38,0.63,0.69}{\textbf{\textit{{#1}}}}}
    
    
    % Define a nice break command that doesn't care if a line doesn't already
    % exist.
    \def\br{\hspace*{\fill} \\* }
    % Math Jax compatability definitions
    \def\gt{>}
    \def\lt{<}
    % Document parameters
    \title{openmp\_mpi\_k-means}
    
    
    

    % Pygments definitions
    
\makeatletter
\def\PY@reset{\let\PY@it=\relax \let\PY@bf=\relax%
    \let\PY@ul=\relax \let\PY@tc=\relax%
    \let\PY@bc=\relax \let\PY@ff=\relax}
\def\PY@tok#1{\csname PY@tok@#1\endcsname}
\def\PY@toks#1+{\ifx\relax#1\empty\else%
    \PY@tok{#1}\expandafter\PY@toks\fi}
\def\PY@do#1{\PY@bc{\PY@tc{\PY@ul{%
    \PY@it{\PY@bf{\PY@ff{#1}}}}}}}
\def\PY#1#2{\PY@reset\PY@toks#1+\relax+\PY@do{#2}}

\expandafter\def\csname PY@tok@gu\endcsname{\let\PY@bf=\textbf\def\PY@tc##1{\textcolor[rgb]{0.50,0.00,0.50}{##1}}}
\expandafter\def\csname PY@tok@sa\endcsname{\def\PY@tc##1{\textcolor[rgb]{0.73,0.13,0.13}{##1}}}
\expandafter\def\csname PY@tok@ss\endcsname{\def\PY@tc##1{\textcolor[rgb]{0.10,0.09,0.49}{##1}}}
\expandafter\def\csname PY@tok@s2\endcsname{\def\PY@tc##1{\textcolor[rgb]{0.73,0.13,0.13}{##1}}}
\expandafter\def\csname PY@tok@kp\endcsname{\def\PY@tc##1{\textcolor[rgb]{0.00,0.50,0.00}{##1}}}
\expandafter\def\csname PY@tok@nd\endcsname{\def\PY@tc##1{\textcolor[rgb]{0.67,0.13,1.00}{##1}}}
\expandafter\def\csname PY@tok@nc\endcsname{\let\PY@bf=\textbf\def\PY@tc##1{\textcolor[rgb]{0.00,0.00,1.00}{##1}}}
\expandafter\def\csname PY@tok@nl\endcsname{\def\PY@tc##1{\textcolor[rgb]{0.63,0.63,0.00}{##1}}}
\expandafter\def\csname PY@tok@k\endcsname{\let\PY@bf=\textbf\def\PY@tc##1{\textcolor[rgb]{0.00,0.50,0.00}{##1}}}
\expandafter\def\csname PY@tok@il\endcsname{\def\PY@tc##1{\textcolor[rgb]{0.40,0.40,0.40}{##1}}}
\expandafter\def\csname PY@tok@nb\endcsname{\def\PY@tc##1{\textcolor[rgb]{0.00,0.50,0.00}{##1}}}
\expandafter\def\csname PY@tok@nf\endcsname{\def\PY@tc##1{\textcolor[rgb]{0.00,0.00,1.00}{##1}}}
\expandafter\def\csname PY@tok@no\endcsname{\def\PY@tc##1{\textcolor[rgb]{0.53,0.00,0.00}{##1}}}
\expandafter\def\csname PY@tok@nt\endcsname{\let\PY@bf=\textbf\def\PY@tc##1{\textcolor[rgb]{0.00,0.50,0.00}{##1}}}
\expandafter\def\csname PY@tok@vi\endcsname{\def\PY@tc##1{\textcolor[rgb]{0.10,0.09,0.49}{##1}}}
\expandafter\def\csname PY@tok@mh\endcsname{\def\PY@tc##1{\textcolor[rgb]{0.40,0.40,0.40}{##1}}}
\expandafter\def\csname PY@tok@o\endcsname{\def\PY@tc##1{\textcolor[rgb]{0.40,0.40,0.40}{##1}}}
\expandafter\def\csname PY@tok@mb\endcsname{\def\PY@tc##1{\textcolor[rgb]{0.40,0.40,0.40}{##1}}}
\expandafter\def\csname PY@tok@err\endcsname{\def\PY@bc##1{\setlength{\fboxsep}{0pt}\fcolorbox[rgb]{1.00,0.00,0.00}{1,1,1}{\strut ##1}}}
\expandafter\def\csname PY@tok@ch\endcsname{\let\PY@it=\textit\def\PY@tc##1{\textcolor[rgb]{0.25,0.50,0.50}{##1}}}
\expandafter\def\csname PY@tok@dl\endcsname{\def\PY@tc##1{\textcolor[rgb]{0.73,0.13,0.13}{##1}}}
\expandafter\def\csname PY@tok@kc\endcsname{\let\PY@bf=\textbf\def\PY@tc##1{\textcolor[rgb]{0.00,0.50,0.00}{##1}}}
\expandafter\def\csname PY@tok@gh\endcsname{\let\PY@bf=\textbf\def\PY@tc##1{\textcolor[rgb]{0.00,0.00,0.50}{##1}}}
\expandafter\def\csname PY@tok@cs\endcsname{\let\PY@it=\textit\def\PY@tc##1{\textcolor[rgb]{0.25,0.50,0.50}{##1}}}
\expandafter\def\csname PY@tok@sx\endcsname{\def\PY@tc##1{\textcolor[rgb]{0.00,0.50,0.00}{##1}}}
\expandafter\def\csname PY@tok@kt\endcsname{\def\PY@tc##1{\textcolor[rgb]{0.69,0.00,0.25}{##1}}}
\expandafter\def\csname PY@tok@cpf\endcsname{\let\PY@it=\textit\def\PY@tc##1{\textcolor[rgb]{0.25,0.50,0.50}{##1}}}
\expandafter\def\csname PY@tok@c\endcsname{\let\PY@it=\textit\def\PY@tc##1{\textcolor[rgb]{0.25,0.50,0.50}{##1}}}
\expandafter\def\csname PY@tok@s\endcsname{\def\PY@tc##1{\textcolor[rgb]{0.73,0.13,0.13}{##1}}}
\expandafter\def\csname PY@tok@w\endcsname{\def\PY@tc##1{\textcolor[rgb]{0.73,0.73,0.73}{##1}}}
\expandafter\def\csname PY@tok@kd\endcsname{\let\PY@bf=\textbf\def\PY@tc##1{\textcolor[rgb]{0.00,0.50,0.00}{##1}}}
\expandafter\def\csname PY@tok@gi\endcsname{\def\PY@tc##1{\textcolor[rgb]{0.00,0.63,0.00}{##1}}}
\expandafter\def\csname PY@tok@mf\endcsname{\def\PY@tc##1{\textcolor[rgb]{0.40,0.40,0.40}{##1}}}
\expandafter\def\csname PY@tok@mi\endcsname{\def\PY@tc##1{\textcolor[rgb]{0.40,0.40,0.40}{##1}}}
\expandafter\def\csname PY@tok@kn\endcsname{\let\PY@bf=\textbf\def\PY@tc##1{\textcolor[rgb]{0.00,0.50,0.00}{##1}}}
\expandafter\def\csname PY@tok@na\endcsname{\def\PY@tc##1{\textcolor[rgb]{0.49,0.56,0.16}{##1}}}
\expandafter\def\csname PY@tok@gr\endcsname{\def\PY@tc##1{\textcolor[rgb]{1.00,0.00,0.00}{##1}}}
\expandafter\def\csname PY@tok@s1\endcsname{\def\PY@tc##1{\textcolor[rgb]{0.73,0.13,0.13}{##1}}}
\expandafter\def\csname PY@tok@si\endcsname{\let\PY@bf=\textbf\def\PY@tc##1{\textcolor[rgb]{0.73,0.40,0.53}{##1}}}
\expandafter\def\csname PY@tok@sh\endcsname{\def\PY@tc##1{\textcolor[rgb]{0.73,0.13,0.13}{##1}}}
\expandafter\def\csname PY@tok@sr\endcsname{\def\PY@tc##1{\textcolor[rgb]{0.73,0.40,0.53}{##1}}}
\expandafter\def\csname PY@tok@cp\endcsname{\def\PY@tc##1{\textcolor[rgb]{0.74,0.48,0.00}{##1}}}
\expandafter\def\csname PY@tok@go\endcsname{\def\PY@tc##1{\textcolor[rgb]{0.53,0.53,0.53}{##1}}}
\expandafter\def\csname PY@tok@se\endcsname{\let\PY@bf=\textbf\def\PY@tc##1{\textcolor[rgb]{0.73,0.40,0.13}{##1}}}
\expandafter\def\csname PY@tok@fm\endcsname{\def\PY@tc##1{\textcolor[rgb]{0.00,0.00,1.00}{##1}}}
\expandafter\def\csname PY@tok@gt\endcsname{\def\PY@tc##1{\textcolor[rgb]{0.00,0.27,0.87}{##1}}}
\expandafter\def\csname PY@tok@sc\endcsname{\def\PY@tc##1{\textcolor[rgb]{0.73,0.13,0.13}{##1}}}
\expandafter\def\csname PY@tok@mo\endcsname{\def\PY@tc##1{\textcolor[rgb]{0.40,0.40,0.40}{##1}}}
\expandafter\def\csname PY@tok@c1\endcsname{\let\PY@it=\textit\def\PY@tc##1{\textcolor[rgb]{0.25,0.50,0.50}{##1}}}
\expandafter\def\csname PY@tok@cm\endcsname{\let\PY@it=\textit\def\PY@tc##1{\textcolor[rgb]{0.25,0.50,0.50}{##1}}}
\expandafter\def\csname PY@tok@gd\endcsname{\def\PY@tc##1{\textcolor[rgb]{0.63,0.00,0.00}{##1}}}
\expandafter\def\csname PY@tok@sb\endcsname{\def\PY@tc##1{\textcolor[rgb]{0.73,0.13,0.13}{##1}}}
\expandafter\def\csname PY@tok@vm\endcsname{\def\PY@tc##1{\textcolor[rgb]{0.10,0.09,0.49}{##1}}}
\expandafter\def\csname PY@tok@ow\endcsname{\let\PY@bf=\textbf\def\PY@tc##1{\textcolor[rgb]{0.67,0.13,1.00}{##1}}}
\expandafter\def\csname PY@tok@nn\endcsname{\let\PY@bf=\textbf\def\PY@tc##1{\textcolor[rgb]{0.00,0.00,1.00}{##1}}}
\expandafter\def\csname PY@tok@vc\endcsname{\def\PY@tc##1{\textcolor[rgb]{0.10,0.09,0.49}{##1}}}
\expandafter\def\csname PY@tok@vg\endcsname{\def\PY@tc##1{\textcolor[rgb]{0.10,0.09,0.49}{##1}}}
\expandafter\def\csname PY@tok@ne\endcsname{\let\PY@bf=\textbf\def\PY@tc##1{\textcolor[rgb]{0.82,0.25,0.23}{##1}}}
\expandafter\def\csname PY@tok@gs\endcsname{\let\PY@bf=\textbf}
\expandafter\def\csname PY@tok@ni\endcsname{\let\PY@bf=\textbf\def\PY@tc##1{\textcolor[rgb]{0.60,0.60,0.60}{##1}}}
\expandafter\def\csname PY@tok@nv\endcsname{\def\PY@tc##1{\textcolor[rgb]{0.10,0.09,0.49}{##1}}}
\expandafter\def\csname PY@tok@ge\endcsname{\let\PY@it=\textit}
\expandafter\def\csname PY@tok@m\endcsname{\def\PY@tc##1{\textcolor[rgb]{0.40,0.40,0.40}{##1}}}
\expandafter\def\csname PY@tok@kr\endcsname{\let\PY@bf=\textbf\def\PY@tc##1{\textcolor[rgb]{0.00,0.50,0.00}{##1}}}
\expandafter\def\csname PY@tok@gp\endcsname{\let\PY@bf=\textbf\def\PY@tc##1{\textcolor[rgb]{0.00,0.00,0.50}{##1}}}
\expandafter\def\csname PY@tok@sd\endcsname{\let\PY@it=\textit\def\PY@tc##1{\textcolor[rgb]{0.73,0.13,0.13}{##1}}}
\expandafter\def\csname PY@tok@bp\endcsname{\def\PY@tc##1{\textcolor[rgb]{0.00,0.50,0.00}{##1}}}

\def\PYZbs{\char`\\}
\def\PYZus{\char`\_}
\def\PYZob{\char`\{}
\def\PYZcb{\char`\}}
\def\PYZca{\char`\^}
\def\PYZam{\char`\&}
\def\PYZlt{\char`\<}
\def\PYZgt{\char`\>}
\def\PYZsh{\char`\#}
\def\PYZpc{\char`\%}
\def\PYZdl{\char`\$}
\def\PYZhy{\char`\-}
\def\PYZsq{\char`\'}
\def\PYZdq{\char`\"}
\def\PYZti{\char`\~}
% for compatibility with earlier versions
\def\PYZat{@}
\def\PYZlb{[}
\def\PYZrb{]}
\makeatother


    % Exact colors from NB
    \definecolor{incolor}{rgb}{0.0, 0.0, 0.5}
    \definecolor{outcolor}{rgb}{0.545, 0.0, 0.0}



    
    % Prevent overflowing lines due to hard-to-break entities
    \sloppy 
    % Setup hyperref package
    \hypersetup{
      breaklinks=true,  % so long urls are correctly broken across lines
      colorlinks=true,
      urlcolor=urlcolor,
      linkcolor=linkcolor,
      citecolor=citecolor,
      }
    % Slightly bigger margins than the latex defaults
    
    \geometry{verbose,tmargin=1in,bmargin=1in,lmargin=1in,rmargin=1in}
    
    

    \begin{document}
    
    
    \maketitle
    
    

    
    \hypertarget{k-means-clustering-with-openmp-and-mpi}{%
\section{K-Means Clustering with OpenMP and
MPI}\label{k-means-clustering-with-openmp-and-mpi}}

\hypertarget{implement-the-k-means-clustering-algorithm-in-openmpmpi-trying-to-maximize-the-performance-reduce-the-execution-time-by-carefully-exploiting-the-resources-within-one-computing-node-with-multiple-processing-cores-openmp-and-across-computing-nodes-mpi.}{%
\subsubsection{Implement the k-means clustering algorithm in OpenMP/MPI,
trying to maximize the performance (reduce the execution time) by
carefully exploiting the resources within one computing node with
multiple processing cores (OpenMP) and across computing nodes
(MPI).}\label{implement-the-k-means-clustering-algorithm-in-openmpmpi-trying-to-maximize-the-performance-reduce-the-execution-time-by-carefully-exploiting-the-resources-within-one-computing-node-with-multiple-processing-cores-openmp-and-across-computing-nodes-mpi.}}

Optional: implement the same algorithm in Apache Flink and compare the
performance of the two implementations (processing time and scalability)
under various workloads.

\begin{center}\rule{0.5\linewidth}{\linethickness}\end{center}

    \begin{Verbatim}[commandchars=\\\{\}]
{\color{incolor}In [{\color{incolor}17}]:} \PY{k+kn}{import} \PY{n+nn}{csv}
         \PY{k+kn}{import} \PY{n+nn}{random}
         \PY{k+kn}{import} \PY{n+nn}{sys}
         
         \PY{k+kn}{import} \PY{n+nn}{numpy}
         \PY{k+kn}{import} \PY{n+nn}{os}  \PY{c+c1}{\PYZsh{} We need this module}
         \PY{k+kn}{import} \PY{n+nn}{matplotlib}\PY{n+nn}{.}\PY{n+nn}{pyplot} \PY{k}{as} \PY{n+nn}{plt}
         \PY{k+kn}{import} \PY{n+nn}{seaborn} \PY{k}{as} \PY{n+nn}{sns}\PY{p}{;}
         \PY{k+kn}{from} \PY{n+nn}{sklearn}\PY{n+nn}{.}\PY{n+nn}{datasets}\PY{n+nn}{.}\PY{n+nn}{samples\PYZus{}generator} \PY{k}{import} \PY{n}{make\PYZus{}blobs}
         \PY{n}{sns}\PY{o}{.}\PY{n}{set}\PY{p}{(}\PY{p}{)}  \PY{c+c1}{\PYZsh{} for plot styling}
         
         \PY{c+c1}{\PYZsh{}Samples}
         \PY{n}{N\PYZus{}SAMPLES} \PY{o}{=} \PY{l+m+mi}{100000}
         
         \PY{c+c1}{\PYZsh{} Get path of the current dir, then use it to create paths:}
         \PY{n}{CURRENT\PYZus{}DIR} \PY{o}{=} \PY{n}{os}\PY{o}{.}\PY{n}{path}\PY{o}{.}\PY{n}{dirname}\PY{p}{(}\PY{l+s+s2}{\PYZdq{}}\PY{l+s+s2}{\PYZus{}\PYZus{}file\PYZus{}\PYZus{}}\PY{l+s+s2}{\PYZdq{}}\PY{p}{)}
         \PY{n}{file\PYZus{}path} \PY{o}{=} \PY{n}{os}\PY{o}{.}\PY{n}{path}\PY{o}{.}\PY{n}{join}\PY{p}{(}\PY{n}{CURRENT\PYZus{}DIR}\PY{p}{,} \PY{l+s+s1}{\PYZsq{}}\PY{l+s+s1}{dataset\PYZus{}display/dataset.csv}\PY{l+s+s1}{\PYZsq{}}\PY{p}{)}
         \PY{n}{initial\PYZus{}dataset\PYZus{}path} \PY{o}{=} \PY{n}{os}\PY{o}{.}\PY{n}{path}\PY{o}{.}\PY{n}{join}\PY{p}{(}\PY{n}{CURRENT\PYZus{}DIR}\PY{p}{,} \PY{l+s+s1}{\PYZsq{}}\PY{l+s+s1}{dataset\PYZus{}display/initialdataset.csv}\PY{l+s+s1}{\PYZsq{}}\PY{p}{)}
         \PY{n}{initial\PYZus{}centroids\PYZus{}path} \PY{o}{=} \PY{n}{os}\PY{o}{.}\PY{n}{path}\PY{o}{.}\PY{n}{join}\PY{p}{(}\PY{n}{CURRENT\PYZus{}DIR}\PY{p}{,} \PY{l+s+s1}{\PYZsq{}}\PY{l+s+s1}{dataset\PYZus{}display/initialcentroids.csv}\PY{l+s+s1}{\PYZsq{}}\PY{p}{)}
         \PY{n}{new\PYZus{}dataset\PYZus{}path} \PY{o}{=} \PY{n}{os}\PY{o}{.}\PY{n}{path}\PY{o}{.}\PY{n}{join}\PY{p}{(}\PY{n}{CURRENT\PYZus{}DIR}\PY{p}{,} \PY{l+s+s1}{\PYZsq{}}\PY{l+s+s1}{dataset\PYZus{}display/newdataset.csv}\PY{l+s+s1}{\PYZsq{}}\PY{p}{)}
         \PY{n}{new\PYZus{}centroids\PYZus{}path} \PY{o}{=} \PY{n}{os}\PY{o}{.}\PY{n}{path}\PY{o}{.}\PY{n}{join}\PY{p}{(}\PY{n}{CURRENT\PYZus{}DIR}\PY{p}{,} \PY{l+s+s1}{\PYZsq{}}\PY{l+s+s1}{dataset\PYZus{}display/newcentroids.csv}\PY{l+s+s1}{\PYZsq{}}\PY{p}{)}
         \PY{n}{objfun\PYZus{}path} \PY{o}{=} \PY{n}{os}\PY{o}{.}\PY{n}{path}\PY{o}{.}\PY{n}{join}\PY{p}{(}\PY{n}{CURRENT\PYZus{}DIR}\PY{p}{,} \PY{l+s+s1}{\PYZsq{}}\PY{l+s+s1}{dataset\PYZus{}display/objfun.csv}\PY{l+s+s1}{\PYZsq{}}\PY{p}{)}
         \PY{n}{exectimes\PYZus{}path} \PY{o}{=} \PY{n}{os}\PY{o}{.}\PY{n}{path}\PY{o}{.}\PY{n}{join}\PY{p}{(}\PY{n}{CURRENT\PYZus{}DIR}\PY{p}{,} \PY{l+s+s1}{\PYZsq{}}\PY{l+s+s1}{dataset\PYZus{}display/exectimes.csv}\PY{l+s+s1}{\PYZsq{}}\PY{p}{)}
         \PY{n}{results\PYZus{}path} \PY{o}{=} \PY{n}{os}\PY{o}{.}\PY{n}{path}\PY{o}{.}\PY{n}{join}\PY{p}{(}\PY{n}{CURRENT\PYZus{}DIR}\PY{p}{,} \PY{l+s+s1}{\PYZsq{}}\PY{l+s+s1}{dataset\PYZus{}display/results.csv}\PY{l+s+s1}{\PYZsq{}}\PY{p}{)}
         
         \PY{n}{fig\PYZus{}width} \PY{o}{=} \PY{l+m+mi}{12}
         \PY{n}{fig\PYZus{}height} \PY{o}{=} \PY{l+m+mi}{10}
\end{Verbatim}


    \hypertarget{initial-dataset}{%
\section{Initial Dataset}\label{initial-dataset}}

    \hypertarget{display-the-points-in-the-initial-dataset}{%
\subsection{Display the points in the Initial
Dataset:}\label{display-the-points-in-the-initial-dataset}}

    We have a Dataset of 100000 points, randomly generated by a custom
Python script. These points are already somehow divided in clusters so
that we can more easily test the K-Means Clustering algorithm.

    \begin{Verbatim}[commandchars=\\\{\}]
{\color{incolor}In [{\color{incolor}2}]:} \PY{n+nb}{print}\PY{p}{(}\PY{l+s+s1}{\PYZsq{}}\PY{l+s+s1}{X,Y}\PY{l+s+s1}{\PYZsq{}}\PY{p}{)}
        \PY{k}{with} \PY{n+nb}{open}\PY{p}{(}\PY{n}{initial\PYZus{}dataset\PYZus{}path}\PY{p}{)} \PY{k}{as} \PY{n}{csvfile}\PY{p}{:}
            \PY{n}{reader} \PY{o}{=} \PY{n}{csv}\PY{o}{.}\PY{n}{DictReader}\PY{p}{(}\PY{n}{csvfile}\PY{p}{)}
            \PY{k}{for} \PY{n}{i}\PY{p}{,}\PY{n}{row} \PY{o+ow}{in} \PY{n+nb}{enumerate}\PY{p}{(}\PY{n}{reader}\PY{p}{)}\PY{p}{:}
                \PY{n+nb}{print}\PY{p}{(}\PY{n}{row}\PY{p}{[}\PY{l+s+s1}{\PYZsq{}}\PY{l+s+s1}{X}\PY{l+s+s1}{\PYZsq{}}\PY{p}{]}\PY{p}{,} \PY{n}{row}\PY{p}{[}\PY{l+s+s1}{\PYZsq{}}\PY{l+s+s1}{Y}\PY{l+s+s1}{\PYZsq{}}\PY{p}{]}\PY{p}{)}
                \PY{k}{if}\PY{p}{(}\PY{n}{i} \PY{o}{\PYZgt{}}\PY{o}{=} \PY{l+m+mi}{10}\PY{p}{)}\PY{p}{:}
                    \PY{k}{break}
                    
        \PY{n+nb}{print}\PY{p}{(}\PY{l+s+s2}{\PYZdq{}}\PY{l+s+s2}{Total points: }\PY{l+s+si}{\PYZpc{}s}\PY{l+s+s2}{\PYZdq{}} \PY{o}{\PYZpc{}} \PY{n+nb}{format}\PY{p}{(}\PY{n}{N\PYZus{}SAMPLES}\PY{p}{,}\PY{l+s+s1}{\PYZsq{}}\PY{l+s+s1}{d}\PY{l+s+s1}{\PYZsq{}}\PY{p}{)}\PY{p}{)}
\end{Verbatim}


    \begin{Verbatim}[commandchars=\\\{\}]
X,Y
4.762006 -7.700549
4.490764 -8.202433
-1.384802 2.850730
-1.411994 -6.885759
0.045029 1.578344
-7.419469 9.258043
-6.915910 9.343622
-7.663262 8.672174
-2.618907 -6.364270
-1.369107 -8.103168
-7.965163 9.759896
Total points: 100000

    \end{Verbatim}

    \hypertarget{and-now-we-plot-the-initial-dataset}{%
\subsection{And now we plot the initial
dataset:}\label{and-now-we-plot-the-initial-dataset}}

    By plotting the Dataset, we can see that is indeed form by 4 clusters
easily observable. We'll later test the K-Means Clustering algorithm and
see if it can also individuate the clusters.

    \begin{Verbatim}[commandchars=\\\{\}]
{\color{incolor}In [{\color{incolor}3}]:} \PY{n}{x} \PY{o}{=} \PY{n}{numpy}\PY{o}{.}\PY{n}{zeros}\PY{p}{(}\PY{n}{N\PYZus{}SAMPLES}\PY{p}{)}
        \PY{n}{y} \PY{o}{=} \PY{n}{numpy}\PY{o}{.}\PY{n}{zeros}\PY{p}{(}\PY{n}{N\PYZus{}SAMPLES}\PY{p}{)}
        
        \PY{c+c1}{\PYZsh{}Read the dataset from the CVS file}
        \PY{k}{with} \PY{n+nb}{open}\PY{p}{(}\PY{n}{initial\PYZus{}dataset\PYZus{}path}\PY{p}{)} \PY{k}{as} \PY{n}{csvfile}\PY{p}{:}
            \PY{n}{reader} \PY{o}{=} \PY{n}{csv}\PY{o}{.}\PY{n}{DictReader}\PY{p}{(}\PY{n}{csvfile}\PY{p}{)}
            \PY{n}{i} \PY{o}{=} \PY{l+m+mi}{0}
            \PY{k}{for} \PY{n}{row} \PY{o+ow}{in} \PY{n}{reader}\PY{p}{:}
                \PY{n}{x}\PY{p}{[}\PY{n}{i}\PY{p}{]} \PY{o}{=} \PY{n}{row}\PY{p}{[}\PY{l+s+s1}{\PYZsq{}}\PY{l+s+s1}{X}\PY{l+s+s1}{\PYZsq{}}\PY{p}{]}
                \PY{n}{y}\PY{p}{[}\PY{n}{i}\PY{p}{]} \PY{o}{=} \PY{n}{row}\PY{p}{[}\PY{l+s+s1}{\PYZsq{}}\PY{l+s+s1}{Y}\PY{l+s+s1}{\PYZsq{}}\PY{p}{]}
                \PY{c+c1}{\PYZsh{}print(x[i], y[i])}
                \PY{n}{i}\PY{o}{=}\PY{n}{i}\PY{o}{+}\PY{l+m+mi}{1}
        
        \PY{c+c1}{\PYZsh{}Plot the read dataset}
        \PY{n}{plt}\PY{o}{.}\PY{n}{figure}\PY{p}{(}\PY{n}{figsize}\PY{o}{=}\PY{p}{(}\PY{n}{fig\PYZus{}width}\PY{p}{,} \PY{n}{fig\PYZus{}height}\PY{p}{)}\PY{p}{,} \PY{n}{dpi}\PY{o}{=} \PY{l+m+mi}{80}\PY{p}{,} \PY{n}{facecolor}\PY{o}{=}\PY{l+s+s1}{\PYZsq{}}\PY{l+s+s1}{w}\PY{l+s+s1}{\PYZsq{}}\PY{p}{,} \PY{n}{edgecolor}\PY{o}{=}\PY{l+s+s1}{\PYZsq{}}\PY{l+s+s1}{k}\PY{l+s+s1}{\PYZsq{}}\PY{p}{)}
        \PY{n}{plt}\PY{o}{.}\PY{n}{scatter}\PY{p}{(}\PY{n}{x}\PY{p}{[}\PY{p}{:}\PY{p}{]}\PY{p}{,} \PY{n}{y}\PY{p}{[}\PY{p}{:}\PY{p}{]}\PY{p}{,} \PY{n}{s}\PY{o}{=}\PY{l+m+mi}{1}\PY{p}{)}
        \PY{n}{plt}\PY{o}{.}\PY{n}{show}\PY{p}{(}\PY{p}{)}
\end{Verbatim}


    \begin{center}
    \adjustimage{max size={0.9\linewidth}{0.9\paperheight}}{output_8_0.png}
    \end{center}
    { \hspace*{\fill} \\}
    
    \hypertarget{initial-centroids}{%
\subsection{Initial Centroids}\label{initial-centroids}}

    The Initial Centroids are randomply placed at runtime, however a
centroids will be recreated if it is too close to another centroid. The
initial placing of the centroids is of fundamental importance to the
result of the K-Means Clustering algorithm execution on a given dataset.
Lets print the centroids:

    \begin{Verbatim}[commandchars=\\\{\}]
{\color{incolor}In [{\color{incolor}4}]:} \PY{n+nb}{print}\PY{p}{(}\PY{l+s+s1}{\PYZsq{}}\PY{l+s+s1}{Cluster,X,Y}\PY{l+s+s1}{\PYZsq{}}\PY{p}{)}
        \PY{n}{centroids} \PY{o}{=} \PY{l+m+mi}{0}
        \PY{k}{with} \PY{n+nb}{open}\PY{p}{(}\PY{n}{initial\PYZus{}centroids\PYZus{}path}\PY{p}{)} \PY{k}{as} \PY{n}{csvfile}\PY{p}{:}
            \PY{n}{reader} \PY{o}{=} \PY{n}{csv}\PY{o}{.}\PY{n}{DictReader}\PY{p}{(}\PY{n}{csvfile}\PY{p}{)}
            \PY{k}{for} \PY{n}{row} \PY{o+ow}{in} \PY{n}{reader}\PY{p}{:}
                \PY{n+nb}{print}\PY{p}{(}\PY{n}{centroids}\PY{p}{,} \PY{n}{row}\PY{p}{[}\PY{l+s+s1}{\PYZsq{}}\PY{l+s+s1}{X}\PY{l+s+s1}{\PYZsq{}}\PY{p}{]}\PY{p}{,} \PY{n}{row}\PY{p}{[}\PY{l+s+s1}{\PYZsq{}}\PY{l+s+s1}{Y}\PY{l+s+s1}{\PYZsq{}}\PY{p}{]}\PY{p}{)}
                \PY{n}{centroids} \PY{o}{=} \PY{n}{centroids} \PY{o}{+} \PY{l+m+mi}{1}
        \PY{n+nb}{print}\PY{p}{(}\PY{l+s+s2}{\PYZdq{}}\PY{l+s+s2}{Total centroids: }\PY{l+s+si}{\PYZpc{}d}\PY{l+s+s2}{\PYZdq{}} \PY{o}{\PYZpc{}} \PY{n}{centroids}\PY{p}{)}
\end{Verbatim}


    \begin{Verbatim}[commandchars=\\\{\}]
Cluster,X,Y
0 4.066931 8.023582
1 -1.933069 -5.976418
2 1.066931 5.023582
3 5.066931 0.023582
Total centroids: 4

    \end{Verbatim}

    And now we plot them:

    \begin{Verbatim}[commandchars=\\\{\}]
{\color{incolor}In [{\color{incolor}5}]:} \PY{n}{x} \PY{o}{=} \PY{n}{numpy}\PY{o}{.}\PY{n}{zeros}\PY{p}{(}\PY{n}{centroids}\PY{p}{)}
        \PY{n}{y} \PY{o}{=} \PY{n}{numpy}\PY{o}{.}\PY{n}{zeros}\PY{p}{(}\PY{n}{centroids}\PY{p}{)}
        
        \PY{c+c1}{\PYZsh{}Read the dataset from the CVS file}
        \PY{k}{with} \PY{n+nb}{open}\PY{p}{(}\PY{n}{initial\PYZus{}centroids\PYZus{}path}\PY{p}{)} \PY{k}{as} \PY{n}{csvfile}\PY{p}{:}
            \PY{n}{reader} \PY{o}{=} \PY{n}{csv}\PY{o}{.}\PY{n}{DictReader}\PY{p}{(}\PY{n}{csvfile}\PY{p}{)}
            \PY{n}{i} \PY{o}{=} \PY{l+m+mi}{0}
            \PY{k}{for} \PY{n}{row} \PY{o+ow}{in} \PY{n}{reader}\PY{p}{:}
                \PY{n}{x}\PY{p}{[}\PY{n}{i}\PY{p}{]} \PY{o}{=} \PY{n}{row}\PY{p}{[}\PY{l+s+s1}{\PYZsq{}}\PY{l+s+s1}{X}\PY{l+s+s1}{\PYZsq{}}\PY{p}{]}
                \PY{n}{y}\PY{p}{[}\PY{n}{i}\PY{p}{]} \PY{o}{=} \PY{n}{row}\PY{p}{[}\PY{l+s+s1}{\PYZsq{}}\PY{l+s+s1}{Y}\PY{l+s+s1}{\PYZsq{}}\PY{p}{]}
                \PY{c+c1}{\PYZsh{}print(x[i], y[i])}
                \PY{n}{i}\PY{o}{=}\PY{n}{i}\PY{o}{+}\PY{l+m+mi}{1}
        
        \PY{c+c1}{\PYZsh{}Plot the read dataset}
        \PY{n}{plt}\PY{o}{.}\PY{n}{figure}\PY{p}{(}\PY{n}{figsize}\PY{o}{=}\PY{p}{(}\PY{n}{fig\PYZus{}width}\PY{p}{,} \PY{n}{fig\PYZus{}height}\PY{p}{)}\PY{p}{,} \PY{n}{dpi}\PY{o}{=} \PY{l+m+mi}{80}\PY{p}{,} \PY{n}{facecolor}\PY{o}{=}\PY{l+s+s1}{\PYZsq{}}\PY{l+s+s1}{w}\PY{l+s+s1}{\PYZsq{}}\PY{p}{,} \PY{n}{edgecolor}\PY{o}{=}\PY{l+s+s1}{\PYZsq{}}\PY{l+s+s1}{k}\PY{l+s+s1}{\PYZsq{}}\PY{p}{)}
        \PY{n}{plt}\PY{o}{.}\PY{n}{scatter}\PY{p}{(}\PY{n}{x}\PY{p}{[}\PY{p}{:}\PY{p}{]}\PY{p}{,} \PY{n}{y}\PY{p}{[}\PY{p}{:}\PY{p}{]}\PY{p}{,} \PY{n}{marker}\PY{o}{=}\PY{l+s+s2}{\PYZdq{}}\PY{l+s+s2}{X}\PY{l+s+s2}{\PYZdq{}}\PY{p}{,} \PY{n}{s}\PY{o}{=}\PY{l+m+mi}{100}\PY{p}{)}
        \PY{n}{plt}\PY{o}{.}\PY{n}{show}\PY{p}{(}\PY{p}{)}
\end{Verbatim}


    \begin{center}
    \adjustimage{max size={0.9\linewidth}{0.9\paperheight}}{output_13_0.png}
    \end{center}
    { \hspace*{\fill} \\}
    
    \hypertarget{plotting-both-dataset-and-centroids}{%
\section{Plotting both Dataset and
Centroids}\label{plotting-both-dataset-and-centroids}}

    We've run the K-Means Clustering Algorithm on the initial dataset and
got new centroids each associated to a cluster. So we now plot the
resulting clusters separated by a random color and their centroids:

    \begin{Verbatim}[commandchars=\\\{\}]
{\color{incolor}In [{\color{incolor}6}]:} \PY{k}{def} \PY{n+nf}{random\PYZus{}color}\PY{p}{(}\PY{p}{)}\PY{p}{:}
            \PY{k}{return} \PY{n}{numpy}\PY{o}{.}\PY{n}{random}\PY{o}{.}\PY{n}{rand}\PY{p}{(}\PY{l+m+mi}{3}\PY{p}{,}\PY{p}{)}
\end{Verbatim}


    \begin{Verbatim}[commandchars=\\\{\}]
{\color{incolor}In [{\color{incolor}7}]:} \PY{n}{x} \PY{o}{=} \PY{n}{numpy}\PY{o}{.}\PY{n}{zeros}\PY{p}{(}\PY{n}{N\PYZus{}SAMPLES}\PY{p}{)}
        \PY{n}{y} \PY{o}{=} \PY{n}{numpy}\PY{o}{.}\PY{n}{zeros}\PY{p}{(}\PY{n}{N\PYZus{}SAMPLES}\PY{p}{)}
        \PY{n}{c} \PY{o}{=} \PY{n}{numpy}\PY{o}{.}\PY{n}{zeros}\PY{p}{(}\PY{n}{N\PYZus{}SAMPLES}\PY{p}{)}
        \PY{n}{cx} \PY{o}{=} \PY{n+nb}{list}\PY{p}{(}\PY{p}{)}
        \PY{n}{cy} \PY{o}{=} \PY{n+nb}{list}\PY{p}{(}\PY{p}{)}
        
        \PY{n}{plt}\PY{o}{.}\PY{n}{figure}\PY{p}{(}\PY{n}{figsize}\PY{o}{=}\PY{p}{(}\PY{n}{fig\PYZus{}width}\PY{p}{,} \PY{n}{fig\PYZus{}height}\PY{p}{)}\PY{p}{,} \PY{n}{dpi}\PY{o}{=} \PY{l+m+mi}{80}\PY{p}{,} \PY{n}{facecolor}\PY{o}{=}\PY{l+s+s1}{\PYZsq{}}\PY{l+s+s1}{w}\PY{l+s+s1}{\PYZsq{}}\PY{p}{,} \PY{n}{edgecolor}\PY{o}{=}\PY{l+s+s1}{\PYZsq{}}\PY{l+s+s1}{k}\PY{l+s+s1}{\PYZsq{}}\PY{p}{)}
        
        \PY{c+c1}{\PYZsh{} Read the new centroids from the CVS file}
        \PY{k}{with} \PY{n+nb}{open}\PY{p}{(}\PY{n}{new\PYZus{}centroids\PYZus{}path}\PY{p}{)} \PY{k}{as} \PY{n}{csvfile}\PY{p}{:}
            \PY{n}{reader} \PY{o}{=} \PY{n}{csv}\PY{o}{.}\PY{n}{DictReader}\PY{p}{(}\PY{n}{csvfile}\PY{p}{)}
            \PY{k}{for} \PY{n}{row} \PY{o+ow}{in} \PY{n}{reader}\PY{p}{:}
                \PY{n}{cx}\PY{o}{.}\PY{n}{append}\PY{p}{(}\PY{n+nb}{float}\PY{p}{(}\PY{n}{row}\PY{p}{[}\PY{l+s+s1}{\PYZsq{}}\PY{l+s+s1}{X}\PY{l+s+s1}{\PYZsq{}}\PY{p}{]}\PY{p}{)}\PY{p}{)}
                \PY{n}{cy}\PY{o}{.}\PY{n}{append}\PY{p}{(}\PY{n+nb}{float}\PY{p}{(}\PY{n}{row}\PY{p}{[}\PY{l+s+s1}{\PYZsq{}}\PY{l+s+s1}{Y}\PY{l+s+s1}{\PYZsq{}}\PY{p}{]}\PY{p}{)}\PY{p}{)}
        
        \PY{c+c1}{\PYZsh{} Read the new dataset from the CVS file}
        \PY{k}{with} \PY{n+nb}{open}\PY{p}{(}\PY{n}{new\PYZus{}dataset\PYZus{}path}\PY{p}{)} \PY{k}{as} \PY{n}{csvfile}\PY{p}{:}
            \PY{n}{reader} \PY{o}{=} \PY{n}{csv}\PY{o}{.}\PY{n}{DictReader}\PY{p}{(}\PY{n}{csvfile}\PY{p}{)}
            \PY{n}{i} \PY{o}{=} \PY{l+m+mi}{0}
            \PY{k}{for} \PY{n}{row} \PY{o+ow}{in} \PY{n}{reader}\PY{p}{:}
                \PY{n}{x}\PY{p}{[}\PY{n}{i}\PY{p}{]} \PY{o}{=} \PY{n}{row}\PY{p}{[}\PY{l+s+s1}{\PYZsq{}}\PY{l+s+s1}{X}\PY{l+s+s1}{\PYZsq{}}\PY{p}{]}
                \PY{n}{y}\PY{p}{[}\PY{n}{i}\PY{p}{]} \PY{o}{=} \PY{n}{row}\PY{p}{[}\PY{l+s+s1}{\PYZsq{}}\PY{l+s+s1}{Y}\PY{l+s+s1}{\PYZsq{}}\PY{p}{]}
                \PY{n}{c}\PY{p}{[}\PY{n}{i}\PY{p}{]} \PY{o}{=} \PY{n}{row}\PY{p}{[}\PY{l+s+s1}{\PYZsq{}}\PY{l+s+s1}{Cluster}\PY{l+s+s1}{\PYZsq{}}\PY{p}{]}
                \PY{c+c1}{\PYZsh{} print(x[i], y[i])}
                \PY{n}{i} \PY{o}{=} \PY{n}{i} \PY{o}{+} \PY{l+m+mi}{1}
        
        \PY{n}{minK} \PY{o}{=} \PY{n}{c}\PY{o}{.}\PY{n}{min}\PY{p}{(}\PY{p}{)}
        \PY{n}{maxK} \PY{o}{=} \PY{n}{c}\PY{o}{.}\PY{n}{max}\PY{p}{(}\PY{p}{)}
        \PY{n}{k} \PY{o}{=} \PY{p}{(}\PY{n+nb}{int}\PY{p}{)}\PY{p}{(}\PY{n}{maxK} \PY{o}{\PYZhy{}} \PY{n}{minK} \PY{o}{+} \PY{l+m+mi}{1}\PY{p}{)}
        
        \PY{c+c1}{\PYZsh{} plot the points for each cluster with a different color}
        \PY{k}{for} \PY{n}{i} \PY{o+ow}{in} \PY{n+nb}{range}\PY{p}{(}\PY{n}{k}\PY{p}{)}\PY{p}{:}
            \PY{n}{x2} \PY{o}{=} \PY{n+nb}{list}\PY{p}{(}\PY{p}{)}
            \PY{n}{y2} \PY{o}{=} \PY{n+nb}{list}\PY{p}{(}\PY{p}{)}
        
            \PY{k}{for} \PY{n}{j} \PY{o+ow}{in} \PY{n+nb}{range}\PY{p}{(}\PY{n}{N\PYZus{}SAMPLES}\PY{p}{)}\PY{p}{:}
                \PY{k}{if} \PY{n}{c}\PY{p}{[}\PY{n}{j}\PY{p}{]} \PY{o}{==} \PY{n}{i}\PY{p}{:}
                    \PY{n}{x2}\PY{o}{.}\PY{n}{append}\PY{p}{(}\PY{n}{x}\PY{p}{[}\PY{n}{j}\PY{p}{]}\PY{p}{)}
                    \PY{n}{y2}\PY{o}{.}\PY{n}{append}\PY{p}{(}\PY{n}{y}\PY{p}{[}\PY{n}{j}\PY{p}{]}\PY{p}{)}
        
            \PY{c+c1}{\PYZsh{} Plot the read dataset}
            \PY{n}{color1} \PY{o}{=} \PY{n}{random\PYZus{}color}\PY{p}{(}\PY{p}{)}
            \PY{n}{color2} \PY{o}{=} \PY{n}{random\PYZus{}color}\PY{p}{(}\PY{p}{)}
            \PY{n}{plt}\PY{o}{.}\PY{n}{scatter}\PY{p}{(}\PY{n}{x2}\PY{p}{[}\PY{p}{:}\PY{p}{]}\PY{p}{,} \PY{n}{y2}\PY{p}{[}\PY{p}{:}\PY{p}{]}\PY{p}{,} \PY{n}{c}\PY{o}{=}\PY{n}{color1}\PY{p}{,} \PY{n}{s}\PY{o}{=}\PY{l+m+mi}{50}\PY{p}{)}
            \PY{n}{plt}\PY{o}{.}\PY{n}{scatter}\PY{p}{(}\PY{n}{cx}\PY{p}{[}\PY{n}{i}\PY{p}{]}\PY{p}{,} \PY{n}{cy}\PY{p}{[}\PY{n}{i}\PY{p}{]}\PY{p}{,} \PY{n}{c}\PY{o}{=}\PY{n}{color1}\PY{p}{,} \PY{n}{marker}\PY{o}{=}\PY{l+s+s2}{\PYZdq{}}\PY{l+s+s2}{X}\PY{l+s+s2}{\PYZdq{}}\PY{p}{,} \PY{n}{edgecolor}\PY{o}{=}\PY{n}{color2}\PY{p}{,} \PY{n}{s}\PY{o}{=}\PY{l+m+mi}{100}\PY{p}{)}
        \PY{n}{plt}\PY{o}{.}\PY{n}{show}\PY{p}{(}\PY{p}{)}
\end{Verbatim}


    \begin{center}
    \adjustimage{max size={0.9\linewidth}{0.9\paperheight}}{output_17_0.png}
    \end{center}
    { \hspace*{\fill} \\}
    
    \hypertarget{running-modes-normal-openmp-mpi}{%
\subsection{Running Modes: Normal, OpenMP,
MPI}\label{running-modes-normal-openmp-mpi}}

    The program written in C actually executes three different versions of
the K-Means Clustering Algorithm. They run sequentially one at a time,
but they use the same initial dataset and the same random pair of
initial centroids. The initial position of the centroids is of vital
importance to the result of the execution so is important that all three
versions use the same set of centroids so that we can have a meaningful
comparison in performance. All three versions will produce the same
result and same objective function value, but they will have different
execution times. We've set the maximum number of cores for the OpenMP
verision (4 on this PC), and 4 parallel processes for the MPI Version.
As expected the OpenMP and MPI version perform much better that the
sequential version, here are the result on this dataset:

    \begin{Verbatim}[commandchars=\\\{\}]
{\color{incolor}In [{\color{incolor}11}]:} \PY{n+nb}{print}\PY{p}{(}\PY{l+s+s1}{\PYZsq{}}\PY{l+s+s1}{Execution Times:}\PY{l+s+se}{\PYZbs{}n}\PY{l+s+s1}{\PYZsq{}}\PY{p}{)}
         \PY{n}{t} \PY{o}{=} \PY{n}{numpy}\PY{o}{.}\PY{n}{zeros}\PY{p}{(}\PY{l+m+mi}{3}\PY{p}{)}
         \PY{k}{with} \PY{n+nb}{open}\PY{p}{(}\PY{n}{exectimes\PYZus{}path}\PY{p}{)} \PY{k}{as} \PY{n}{csvfile}\PY{p}{:}
             \PY{n}{reader} \PY{o}{=} \PY{n}{csv}\PY{o}{.}\PY{n}{DictReader}\PY{p}{(}\PY{n}{csvfile}\PY{p}{)}
             \PY{k}{for} \PY{n}{i}\PY{p}{,}\PY{n}{row} \PY{o+ow}{in} \PY{n+nb}{enumerate}\PY{p}{(}\PY{n}{reader}\PY{p}{)}\PY{p}{:}
                 \PY{n}{t}\PY{p}{[}\PY{n}{i}\PY{p}{]} \PY{o}{=} \PY{n}{row}\PY{p}{[}\PY{l+s+s1}{\PYZsq{}}\PY{l+s+s1}{Time}\PY{l+s+s1}{\PYZsq{}}\PY{p}{]}
                     
         \PY{n+nb}{print}\PY{p}{(}\PY{l+s+s1}{\PYZsq{}}\PY{l+s+s1}{Normal Execution \PYZhy{}\PYZhy{}\PYZgt{} }\PY{l+s+s1}{\PYZsq{}} \PY{o}{+} \PY{n+nb}{str}\PY{p}{(}\PY{n}{t}\PY{p}{[}\PY{l+m+mi}{0}\PY{p}{]}\PY{o}{*}\PY{l+m+mi}{1000}\PY{p}{)} \PY{o}{+} \PY{l+s+s1}{\PYZsq{}}\PY{l+s+s1}{ms}\PY{l+s+s1}{\PYZsq{}}\PY{p}{)}
         \PY{n+nb}{print}\PY{p}{(}\PY{l+s+s1}{\PYZsq{}}\PY{l+s+s1}{OpenMP Execution \PYZhy{}\PYZhy{}\PYZgt{} }\PY{l+s+s1}{\PYZsq{}} \PY{o}{+} \PY{n+nb}{str}\PY{p}{(}\PY{n}{t}\PY{p}{[}\PY{l+m+mi}{1}\PY{p}{]}\PY{o}{*}\PY{l+m+mi}{1000}\PY{p}{)} \PY{o}{+} \PY{l+s+s1}{\PYZsq{}}\PY{l+s+s1}{ms}\PY{l+s+s1}{\PYZsq{}}\PY{p}{)}
         \PY{n+nb}{print}\PY{p}{(}\PY{l+s+s1}{\PYZsq{}}\PY{l+s+s1}{MPI Execution    \PYZhy{}\PYZhy{}\PYZgt{} }\PY{l+s+s1}{\PYZsq{}} \PY{o}{+} \PY{n+nb}{str}\PY{p}{(}\PY{n}{t}\PY{p}{[}\PY{l+m+mi}{2}\PY{p}{]}\PY{o}{*}\PY{l+m+mi}{1000}\PY{p}{)} \PY{o}{+} \PY{l+s+s1}{\PYZsq{}}\PY{l+s+s1}{ms}\PY{l+s+s1}{\PYZsq{}}\PY{p}{)}
\end{Verbatim}


    \begin{Verbatim}[commandchars=\\\{\}]
Execution Times:


Normal Execution --> 117.774ms
OpenMP Execution --> 35.004ms
MPI Execution    --> 60.367999999999995ms

    \end{Verbatim}

    While the value of the Objective Function is:

    \begin{Verbatim}[commandchars=\\\{\}]
{\color{incolor}In [{\color{incolor}16}]:} \PY{n}{objfun} \PY{o}{=} \PY{n}{numpy}\PY{o}{.}\PY{n}{zeros}\PY{p}{(}\PY{l+m+mi}{3}\PY{p}{)}
         \PY{k}{with} \PY{n+nb}{open}\PY{p}{(}\PY{n}{objfun\PYZus{}path}\PY{p}{)} \PY{k}{as} \PY{n}{csvfile}\PY{p}{:}
             \PY{n}{reader} \PY{o}{=} \PY{n}{csv}\PY{o}{.}\PY{n}{DictReader}\PY{p}{(}\PY{n}{csvfile}\PY{p}{)}
             \PY{k}{for} \PY{n}{i}\PY{p}{,}\PY{n}{row} \PY{o+ow}{in} \PY{n+nb}{enumerate}\PY{p}{(}\PY{n}{reader}\PY{p}{)}\PY{p}{:}
                 \PY{n}{objfun}\PY{p}{[}\PY{n}{i}\PY{p}{]} \PY{o}{=} \PY{n}{row}\PY{p}{[}\PY{l+s+s1}{\PYZsq{}}\PY{l+s+s1}{ObjFun}\PY{l+s+s1}{\PYZsq{}}\PY{p}{]}
                     
         \PY{n+nb}{print}\PY{p}{(}\PY{l+s+s1}{\PYZsq{}}\PY{l+s+s1}{ObjectiveFunction value \PYZhy{}\PYZhy{}\PYZgt{} }\PY{l+s+s1}{\PYZsq{}} \PY{o}{+} \PY{n+nb}{str}\PY{p}{(}\PY{n}{objfun}\PY{p}{[}\PY{l+m+mi}{0}\PY{p}{]}\PY{p}{)}\PY{p}{)}
\end{Verbatim}


    \begin{Verbatim}[commandchars=\\\{\}]
ObjectiveFunction value --> 71883.664062

    \end{Verbatim}

    \hypertarget{section}{%
\subsection{------------------------------------------}\label{section}}

    \hypertarget{results}{%
\section{Results}\label{results}}

    \hypertarget{cumulative-execution-results}{%
\subsubsection{Cumulative Execution
Results}\label{cumulative-execution-results}}

By running a bach script we were able cumulate a large quantity of
execution results (more than 1000 executions). For consistency we've run
all three different versions of the K-Means Clustering Algorithm
sequentially one at a time on the same initial dataset and the same
random pair of initial centroids. That way the results of each execution
can be compered with the executions of the other two versions of the
algorithm. For each executions we've stored the number of centroids, the
execution mode(algorithm version), the execution time and the obj
function result.

\hypertarget{relation-between-number-of-centroids-and-obj-function-value}{%
\subsubsection{Relation between number of centroids and obj function
value}\label{relation-between-number-of-centroids-and-obj-function-value}}

Let's average the executions results in the large dataset described
above and plot the relation between the number of centroids and the obj
function value. The execution were made with a number of centroids
between 1 and 10.

    \begin{Verbatim}[commandchars=\\\{\}]
{\color{incolor}In [{\color{incolor}80}]:} \PY{n}{row\PYZus{}count} \PY{o}{=} \PY{l+m+mi}{0}
         \PY{k}{with} \PY{n+nb}{open}\PY{p}{(}\PY{n}{results\PYZus{}path}\PY{p}{)} \PY{k}{as} \PY{n}{csvfile}\PY{p}{:}
             \PY{n}{row\PYZus{}count} \PY{o}{=} \PY{n+nb}{sum}\PY{p}{(}\PY{l+m+mi}{1} \PY{k}{for} \PY{n}{line} \PY{o+ow}{in} \PY{n}{csvfile}\PY{p}{)}
         \PY{n+nb}{print} \PY{p}{(}\PY{l+s+s2}{\PYZdq{}}\PY{l+s+s2}{Rows in the results file: }\PY{l+s+s2}{\PYZdq{}} \PY{o}{+} \PY{n+nb}{str}\PY{p}{(}\PY{n}{row\PYZus{}count}\PY{p}{)}\PY{p}{)}
         \PY{n}{EXECUTIONS} \PY{o}{=} \PY{n+nb}{int}\PY{p}{(}\PY{p}{(}\PY{n}{row\PYZus{}count} \PY{o}{\PYZhy{}} \PY{l+m+mi}{1}\PY{p}{)}\PY{o}{/}\PY{l+m+mi}{3}\PY{p}{)}  
         
         \PY{n}{n\PYZus{}c}      \PY{o}{=} \PY{n}{numpy}\PY{o}{.}\PY{n}{zeros}\PY{p}{(}\PY{n}{EXECUTIONS}\PY{p}{)}
         \PY{n}{openmp\PYZus{}c} \PY{o}{=} \PY{n}{numpy}\PY{o}{.}\PY{n}{zeros}\PY{p}{(}\PY{n}{EXECUTIONS}\PY{p}{)}
         \PY{n}{mpi\PYZus{}c}    \PY{o}{=} \PY{n}{numpy}\PY{o}{.}\PY{n}{zeros}\PY{p}{(}\PY{n}{EXECUTIONS}\PY{p}{)}
         
         \PY{n}{n\PYZus{}t}      \PY{o}{=} \PY{n}{numpy}\PY{o}{.}\PY{n}{zeros}\PY{p}{(}\PY{n}{EXECUTIONS}\PY{p}{)}
         \PY{n}{openmp\PYZus{}t} \PY{o}{=} \PY{n}{numpy}\PY{o}{.}\PY{n}{zeros}\PY{p}{(}\PY{n}{EXECUTIONS}\PY{p}{)}
         \PY{n}{mpi\PYZus{}t}    \PY{o}{=} \PY{n}{numpy}\PY{o}{.}\PY{n}{zeros}\PY{p}{(}\PY{n}{EXECUTIONS}\PY{p}{)}
         
         \PY{n}{n\PYZus{}of}      \PY{o}{=} \PY{n}{numpy}\PY{o}{.}\PY{n}{zeros}\PY{p}{(}\PY{n}{EXECUTIONS}\PY{p}{)}
         \PY{n}{openmp\PYZus{}of} \PY{o}{=} \PY{n}{numpy}\PY{o}{.}\PY{n}{zeros}\PY{p}{(}\PY{n}{EXECUTIONS}\PY{p}{)}
         \PY{n}{mpi\PYZus{}of}    \PY{o}{=} \PY{n}{numpy}\PY{o}{.}\PY{n}{zeros}\PY{p}{(}\PY{n}{EXECUTIONS}\PY{p}{)}
         
         \PY{c+c1}{\PYZsh{} Read the new centroids from the CVS file}
         \PY{k}{with} \PY{n+nb}{open}\PY{p}{(}\PY{n}{results\PYZus{}path}\PY{p}{)} \PY{k}{as} \PY{n}{csvfile}\PY{p}{:}
             \PY{n}{reader} \PY{o}{=} \PY{n}{csv}\PY{o}{.}\PY{n}{DictReader}\PY{p}{(}\PY{n}{csvfile}\PY{p}{)}
             \PY{n}{i1}\PY{p}{,}\PY{n}{i2}\PY{p}{,}\PY{n}{i3}\PY{o}{=}\PY{l+m+mi}{0}\PY{p}{,}\PY{l+m+mi}{0}\PY{p}{,}\PY{l+m+mi}{0}
             \PY{k}{for} \PY{n}{i}\PY{p}{,}\PY{n}{row} \PY{o+ow}{in} \PY{n+nb}{enumerate}\PY{p}{(}\PY{n}{reader}\PY{p}{)}\PY{p}{:}
                 \PY{n}{mode} \PY{o}{=} \PY{n+nb}{int}\PY{p}{(}\PY{n}{row}\PY{p}{[}\PY{l+s+s1}{\PYZsq{}}\PY{l+s+s1}{Mode}\PY{l+s+s1}{\PYZsq{}}\PY{p}{]}\PY{p}{)}
                 \PY{c+c1}{\PYZsh{}print(mode + \PYZdq{}\PYZhy{}\PYZdq{} + str(i))}
                 
                 \PY{c+c1}{\PYZsh{}Normal Execution}
                 \PY{k}{if} \PY{n}{mode} \PY{o}{==} \PY{l+m+mi}{0}\PY{p}{:}
                     \PY{n}{n\PYZus{}c}\PY{p}{[}\PY{n}{i1}\PY{p}{]}  \PY{o}{=} \PY{n+nb}{float}\PY{p}{(}\PY{n}{row}\PY{p}{[}\PY{l+s+s1}{\PYZsq{}}\PY{l+s+s1}{K}\PY{l+s+s1}{\PYZsq{}}\PY{p}{]}\PY{p}{)}
                     \PY{n}{n\PYZus{}t}\PY{p}{[}\PY{n}{i1}\PY{p}{]}  \PY{o}{=} \PY{n+nb}{float}\PY{p}{(}\PY{n}{row}\PY{p}{[}\PY{l+s+s1}{\PYZsq{}}\PY{l+s+s1}{Time}\PY{l+s+s1}{\PYZsq{}}\PY{p}{]}\PY{p}{)}
                     \PY{n}{n\PYZus{}of}\PY{p}{[}\PY{n}{i1}\PY{p}{]} \PY{o}{=} \PY{n+nb}{float}\PY{p}{(}\PY{n}{row}\PY{p}{[}\PY{l+s+s1}{\PYZsq{}}\PY{l+s+s1}{ObjFun}\PY{l+s+s1}{\PYZsq{}}\PY{p}{]}\PY{p}{)}
                     \PY{n}{i1} \PY{o}{=} \PY{n}{i1} \PY{o}{+} \PY{l+m+mi}{1}
                 \PY{c+c1}{\PYZsh{}OpenMP Execution}
                 \PY{k}{elif} \PY{n}{mode} \PY{o}{==} \PY{l+m+mi}{1}\PY{p}{:}
                     \PY{n}{openmp\PYZus{}c}\PY{p}{[}\PY{n}{i2}\PY{p}{]}  \PY{o}{=} \PY{n+nb}{float}\PY{p}{(}\PY{n}{row}\PY{p}{[}\PY{l+s+s1}{\PYZsq{}}\PY{l+s+s1}{K}\PY{l+s+s1}{\PYZsq{}}\PY{p}{]}\PY{p}{)}
                     \PY{n}{openmp\PYZus{}t}\PY{p}{[}\PY{n}{i2}\PY{p}{]}  \PY{o}{=} \PY{n+nb}{float}\PY{p}{(}\PY{n}{row}\PY{p}{[}\PY{l+s+s1}{\PYZsq{}}\PY{l+s+s1}{Time}\PY{l+s+s1}{\PYZsq{}}\PY{p}{]}\PY{p}{)}
                     \PY{n}{openmp\PYZus{}of}\PY{p}{[}\PY{n}{i2}\PY{p}{]} \PY{o}{=} \PY{n+nb}{float}\PY{p}{(}\PY{n}{row}\PY{p}{[}\PY{l+s+s1}{\PYZsq{}}\PY{l+s+s1}{ObjFun}\PY{l+s+s1}{\PYZsq{}}\PY{p}{]}\PY{p}{)}
                     \PY{n}{i2} \PY{o}{=} \PY{n}{i2} \PY{o}{+} \PY{l+m+mi}{1}
                 \PY{c+c1}{\PYZsh{}MPI Execution}
                 \PY{k}{elif} \PY{n}{mode} \PY{o}{==} \PY{l+m+mi}{2}\PY{p}{:}
                     \PY{n}{mpi\PYZus{}c}\PY{p}{[}\PY{n}{i3}\PY{p}{]}  \PY{o}{=} \PY{n+nb}{float}\PY{p}{(}\PY{n}{row}\PY{p}{[}\PY{l+s+s1}{\PYZsq{}}\PY{l+s+s1}{K}\PY{l+s+s1}{\PYZsq{}}\PY{p}{]}\PY{p}{)}
                     \PY{n}{mpi\PYZus{}t}\PY{p}{[}\PY{n}{i3}\PY{p}{]}  \PY{o}{=} \PY{n+nb}{float}\PY{p}{(}\PY{n}{row}\PY{p}{[}\PY{l+s+s1}{\PYZsq{}}\PY{l+s+s1}{Time}\PY{l+s+s1}{\PYZsq{}}\PY{p}{]}\PY{p}{)}
                     \PY{n}{mpi\PYZus{}of}\PY{p}{[}\PY{n}{i3}\PY{p}{]} \PY{o}{=} \PY{n+nb}{float}\PY{p}{(}\PY{n}{row}\PY{p}{[}\PY{l+s+s1}{\PYZsq{}}\PY{l+s+s1}{ObjFun}\PY{l+s+s1}{\PYZsq{}}\PY{p}{]}\PY{p}{)}
                     \PY{n}{i3} \PY{o}{=} \PY{n}{i3} \PY{o}{+} \PY{l+m+mi}{1}
         
         \PY{n}{x\PYZus{}val} \PY{o}{=} \PY{n}{numpy}\PY{o}{.}\PY{n}{zeros}\PY{p}{(}\PY{l+m+mi}{10}\PY{p}{)}
         \PY{n}{y\PYZus{}val} \PY{o}{=} \PY{n}{numpy}\PY{o}{.}\PY{n}{zeros}\PY{p}{(}\PY{l+m+mi}{10}\PY{p}{)}
         
         \PY{c+c1}{\PYZsh{}Calc the mean values}
         \PY{k}{for} \PY{n}{i} \PY{o+ow}{in} \PY{n+nb}{range}\PY{p}{(}\PY{l+m+mi}{0}\PY{p}{,}\PY{l+m+mi}{10}\PY{p}{)}\PY{p}{:}
             \PY{n}{x\PYZus{}val}\PY{p}{[}\PY{n}{i}\PY{p}{]} \PY{o}{=} \PY{n}{i} \PY{o}{+} \PY{l+m+mi}{1}
             \PY{c+c1}{\PYZsh{}print(str(i))}
             
             \PY{c+c1}{\PYZsh{}Iterate all the executions for each centroid number}
             \PY{k}{for} \PY{n}{j} \PY{o+ow}{in} \PY{n+nb}{range}\PY{p}{(}\PY{n}{EXECUTIONS}\PY{p}{)}\PY{p}{:}
                 \PY{k}{if} \PY{n}{n\PYZus{}c}\PY{p}{[}\PY{n}{j}\PY{p}{]} \PY{o}{==} \PY{n}{i}\PY{o}{+}\PY{l+m+mi}{1}\PY{p}{:}
                     \PY{n}{y\PYZus{}val}\PY{p}{[}\PY{n}{i}\PY{p}{]} \PY{o}{=} \PY{n}{y\PYZus{}val}\PY{p}{[}\PY{n}{i}\PY{p}{]} \PY{o}{+} \PY{n}{n\PYZus{}of}\PY{p}{[}\PY{n}{j}\PY{p}{]}
                 
             \PY{c+c1}{\PYZsh{}Obj Function values mean}
             \PY{n}{y\PYZus{}val}\PY{p}{[}\PY{n}{i}\PY{p}{]} \PY{o}{=} \PY{n}{y\PYZus{}val}\PY{p}{[}\PY{n}{i}\PY{p}{]} \PY{o}{/} \PY{n}{EXECUTIONS}
             \PY{c+c1}{\PYZsh{}print(str(y\PYZus{}val[i]))}
             
         \PY{c+c1}{\PYZsh{}Plot the between number of centroids and obj function value}
         \PY{n}{plt}\PY{o}{.}\PY{n}{figure}\PY{p}{(}\PY{n}{figsize}\PY{o}{=}\PY{p}{(}\PY{n}{fig\PYZus{}width}\PY{p}{,} \PY{n}{fig\PYZus{}height}\PY{p}{)}\PY{p}{,} \PY{n}{dpi}\PY{o}{=} \PY{l+m+mi}{80}\PY{p}{,} \PY{n}{facecolor}\PY{o}{=}\PY{l+s+s1}{\PYZsq{}}\PY{l+s+s1}{w}\PY{l+s+s1}{\PYZsq{}}\PY{p}{,} \PY{n}{edgecolor}\PY{o}{=}\PY{l+s+s1}{\PYZsq{}}\PY{l+s+s1}{k}\PY{l+s+s1}{\PYZsq{}}\PY{p}{)}
         \PY{n}{plt}\PY{o}{.}\PY{n}{plot}\PY{p}{(}\PY{n}{x\PYZus{}val}\PY{p}{,}\PY{n}{y\PYZus{}val}\PY{p}{)}
         \PY{n}{plt}\PY{o}{.}\PY{n}{suptitle}\PY{p}{(}\PY{l+s+s1}{\PYZsq{}}\PY{l+s+s1}{Number of Centroids \PYZhy{} Obj. Function Value}\PY{l+s+s1}{\PYZsq{}}\PY{p}{)}
         \PY{n}{plt}\PY{o}{.}\PY{n}{xlabel}\PY{p}{(}\PY{l+s+s1}{\PYZsq{}}\PY{l+s+s1}{N\PYZsh{} Centroids}\PY{l+s+s1}{\PYZsq{}}\PY{p}{)}
         \PY{n}{plt}\PY{o}{.}\PY{n}{ylabel}\PY{p}{(}\PY{l+s+s1}{\PYZsq{}}\PY{l+s+s1}{ObjFun Value}\PY{l+s+s1}{\PYZsq{}}\PY{p}{)}
         \PY{n}{plt}\PY{o}{.}\PY{n}{show}\PY{p}{(}\PY{p}{)}
\end{Verbatim}


    \begin{Verbatim}[commandchars=\\\{\}]
Rows in the results file: 1201

    \end{Verbatim}

    \begin{center}
    \adjustimage{max size={0.9\linewidth}{0.9\paperheight}}{output_26_1.png}
    \end{center}
    { \hspace*{\fill} \\}
    
    \hypertarget{resulting-plot-analisys}{%
\subsubsection{Resulting plot analisys}\label{resulting-plot-analisys}}

Given the Plot above we can observe the way that the Obj. Function
Values(Y-Axis) change based on the Number of Centroids(X-Asis) used. And
we can clearly notice that the Knee value is equal to ``4''. It's the
Knee value because the successive numbers of centroids don't have an
Obj. Function Value that decreases greatly. That means that the
distances in between the data points and the centroids for each cluster
don't decrease much after the value of 4 centroids used. So just by
observing the above plot and chosing the knee value we can conlude that
the Dataset is composed of 4 clusters witch is correct since we are the
ones that generated the datased divided in four clusters in the first
place.

    \hypertarget{relation-between-number-of-centroids-and-the-execution-time}{%
\subsubsection{Relation between Number of Centroids and the Execution
Time}\label{relation-between-number-of-centroids-and-the-execution-time}}

Let's now again average the executions results in the large dataset
described above and plot the relation between the number of centroids
and the Execution times. The execution were made with a number of
centroids between 1 and 10.

Let's also compare the results for each of the three execution modes:
Normal, OpenMP, MPI

    \begin{Verbatim}[commandchars=\\\{\}]
{\color{incolor}In [{\color{incolor}89}]:} \PY{n}{x\PYZus{}valc}       \PY{o}{=} \PY{n}{numpy}\PY{o}{.}\PY{n}{zeros}\PY{p}{(}\PY{l+m+mi}{10}\PY{p}{)}    \PY{c+c1}{\PYZsh{} Centroids}
         \PY{n}{y\PYZus{}val\PYZus{}n\PYZus{}et}   \PY{o}{=} \PY{n}{numpy}\PY{o}{.}\PY{n}{zeros}\PY{p}{(}\PY{l+m+mi}{10}\PY{p}{)}    \PY{c+c1}{\PYZsh{} Normal Execution Time}
         \PY{n}{y\PYZus{}val\PYZus{}mp\PYZus{}et}  \PY{o}{=} \PY{n}{numpy}\PY{o}{.}\PY{n}{zeros}\PY{p}{(}\PY{l+m+mi}{10}\PY{p}{)}    \PY{c+c1}{\PYZsh{} OpenMP Execution Time}
         \PY{n}{y\PYZus{}val\PYZus{}mpi\PYZus{}et} \PY{o}{=} \PY{n}{numpy}\PY{o}{.}\PY{n}{zeros}\PY{p}{(}\PY{l+m+mi}{10}\PY{p}{)}    \PY{c+c1}{\PYZsh{} MPI Execution Time}
         
         \PY{c+c1}{\PYZsh{}Calc the mean values}
         \PY{k}{for} \PY{n}{i} \PY{o+ow}{in} \PY{n+nb}{range}\PY{p}{(}\PY{l+m+mi}{0}\PY{p}{,}\PY{l+m+mi}{10}\PY{p}{)}\PY{p}{:}
             \PY{n}{x\PYZus{}valc}\PY{p}{[}\PY{n}{i}\PY{p}{]} \PY{o}{=} \PY{n}{i} \PY{o}{+} \PY{l+m+mi}{1}
             \PY{c+c1}{\PYZsh{}print(str(i))}
             
             \PY{c+c1}{\PYZsh{}Iterate all the executions for each centroid number}
             \PY{k}{for} \PY{n}{j} \PY{o+ow}{in} \PY{n+nb}{range}\PY{p}{(}\PY{n}{EXECUTIONS}\PY{p}{)}\PY{p}{:}
                 \PY{c+c1}{\PYZsh{} Normal Execution}
                 \PY{k}{if} \PY{n}{n\PYZus{}c}\PY{p}{[}\PY{n}{j}\PY{p}{]} \PY{o}{==} \PY{n}{i}\PY{o}{+}\PY{l+m+mi}{1}\PY{p}{:}
                     \PY{n}{y\PYZus{}val\PYZus{}n\PYZus{}et}\PY{p}{[}\PY{n}{i}\PY{p}{]} \PY{o}{=} \PY{n}{y\PYZus{}val\PYZus{}n\PYZus{}et}\PY{p}{[}\PY{n}{i}\PY{p}{]} \PY{o}{+} \PY{n}{n\PYZus{}t}\PY{p}{[}\PY{n}{j}\PY{p}{]}
                 
                 \PY{c+c1}{\PYZsh{} OpenMP Execution}
                 \PY{k}{if} \PY{n}{openmp\PYZus{}c}\PY{p}{[}\PY{n}{j}\PY{p}{]} \PY{o}{==} \PY{n}{i}\PY{o}{+}\PY{l+m+mi}{1}\PY{p}{:}
                     \PY{n}{y\PYZus{}val\PYZus{}mp\PYZus{}et}\PY{p}{[}\PY{n}{i}\PY{p}{]} \PY{o}{=} \PY{n}{y\PYZus{}val\PYZus{}mp\PYZus{}et}\PY{p}{[}\PY{n}{i}\PY{p}{]} \PY{o}{+} \PY{n}{openmp\PYZus{}t}\PY{p}{[}\PY{n}{j}\PY{p}{]}
                     
                 \PY{c+c1}{\PYZsh{} MPI Execution}
                 \PY{k}{if} \PY{n}{openmp\PYZus{}c}\PY{p}{[}\PY{n}{j}\PY{p}{]} \PY{o}{==} \PY{n}{i}\PY{o}{+}\PY{l+m+mi}{1}\PY{p}{:}
                     \PY{n}{y\PYZus{}val\PYZus{}mpi\PYZus{}et}\PY{p}{[}\PY{n}{i}\PY{p}{]} \PY{o}{=} \PY{n}{y\PYZus{}val\PYZus{}mpi\PYZus{}et}\PY{p}{[}\PY{n}{i}\PY{p}{]} \PY{o}{+} \PY{n}{mpi\PYZus{}t}\PY{p}{[}\PY{n}{j}\PY{p}{]}
                 
             \PY{c+c1}{\PYZsh{}Execution Time values mean}
             \PY{n}{y\PYZus{}val\PYZus{}n\PYZus{}et}\PY{p}{[}\PY{n}{i}\PY{p}{]}   \PY{o}{=} \PY{p}{(}\PY{n}{y\PYZus{}val\PYZus{}n\PYZus{}et}\PY{p}{[}\PY{n}{i}\PY{p}{]} \PY{o}{/} \PY{n}{EXECUTIONS}\PY{p}{)} \PY{o}{*} \PY{l+m+mi}{1000}
             \PY{n}{y\PYZus{}val\PYZus{}mp\PYZus{}et}\PY{p}{[}\PY{n}{i}\PY{p}{]}  \PY{o}{=} \PY{p}{(}\PY{n}{y\PYZus{}val\PYZus{}mp\PYZus{}et}\PY{p}{[}\PY{n}{i}\PY{p}{]} \PY{o}{/} \PY{n}{EXECUTIONS}\PY{p}{)} \PY{o}{*} \PY{l+m+mi}{1000}
             \PY{n}{y\PYZus{}val\PYZus{}mpi\PYZus{}et}\PY{p}{[}\PY{n}{i}\PY{p}{]} \PY{o}{=} \PY{p}{(}\PY{n}{y\PYZus{}val\PYZus{}mpi\PYZus{}et}\PY{p}{[}\PY{n}{i}\PY{p}{]} \PY{o}{/} \PY{n}{EXECUTIONS}\PY{p}{)} \PY{o}{*} \PY{l+m+mi}{1000}
             \PY{c+c1}{\PYZsh{}print(\PYZdq{}n \PYZdq{} + str(y\PYZus{}val\PYZus{}n\PYZus{}et[i]))}
             \PY{c+c1}{\PYZsh{}print(\PYZdq{}mp \PYZdq{} + str(y\PYZus{}val\PYZus{}mp\PYZus{}et[i]))}
             \PY{c+c1}{\PYZsh{}print(\PYZdq{}mpi \PYZdq{} + str(y\PYZus{}val\PYZus{}mpi\PYZus{}et[i]))}
             
         \PY{c+c1}{\PYZsh{}Plot the between number of centroids and obj function value}
         \PY{n}{plt}\PY{o}{.}\PY{n}{figure}\PY{p}{(}\PY{n}{figsize}\PY{o}{=}\PY{p}{(}\PY{n}{fig\PYZus{}width}\PY{p}{,} \PY{n}{fig\PYZus{}height}\PY{p}{)}\PY{p}{,} \PY{n}{dpi}\PY{o}{=} \PY{l+m+mi}{80}\PY{p}{,} \PY{n}{facecolor}\PY{o}{=}\PY{l+s+s1}{\PYZsq{}}\PY{l+s+s1}{w}\PY{l+s+s1}{\PYZsq{}}\PY{p}{,} \PY{n}{edgecolor}\PY{o}{=}\PY{l+s+s1}{\PYZsq{}}\PY{l+s+s1}{k}\PY{l+s+s1}{\PYZsq{}}\PY{p}{)}
         \PY{n}{plt}\PY{o}{.}\PY{n}{plot}\PY{p}{(}\PY{n}{x\PYZus{}valc}\PY{p}{,}\PY{n}{y\PYZus{}val\PYZus{}n\PYZus{}et}\PY{p}{)}
         \PY{n}{plt}\PY{o}{.}\PY{n}{suptitle}\PY{p}{(}\PY{l+s+s1}{\PYZsq{}}\PY{l+s+s1}{Number of Centroids \PYZhy{} Execution Time}\PY{l+s+s1}{\PYZsq{}}\PY{p}{)}
         \PY{n}{plt}\PY{o}{.}\PY{n}{xlabel}\PY{p}{(}\PY{l+s+s1}{\PYZsq{}}\PY{l+s+s1}{N\PYZsh{} Centroids}\PY{l+s+s1}{\PYZsq{}}\PY{p}{)}
         \PY{n}{plt}\PY{o}{.}\PY{n}{ylabel}\PY{p}{(}\PY{l+s+s1}{\PYZsq{}}\PY{l+s+s1}{Exec. Time [ms]}\PY{l+s+s1}{\PYZsq{}}\PY{p}{)}
         \PY{n}{plt}\PY{o}{.}\PY{n}{plot}\PY{p}{(}\PY{n}{x\PYZus{}valc}\PY{p}{,}\PY{n}{y\PYZus{}val\PYZus{}mp\PYZus{}et}\PY{p}{)}
         \PY{n}{plt}\PY{o}{.}\PY{n}{plot}\PY{p}{(}\PY{n}{x\PYZus{}valc}\PY{p}{,}\PY{n}{y\PYZus{}val\PYZus{}mpi\PYZus{}et}\PY{p}{)}
         \PY{n}{plt}\PY{o}{.}\PY{n}{show}\PY{p}{(}\PY{p}{)}
\end{Verbatim}


    \begin{center}
    \adjustimage{max size={0.9\linewidth}{0.9\paperheight}}{output_29_0.png}
    \end{center}
    { \hspace*{\fill} \\}
    
    \hypertarget{resulting-plot-analisys}{%
\subsubsection{Resulting plot analisys}\label{resulting-plot-analisys}}

On the above plot we can observe the Execution Times(Y-Axis) for each
execution mode base on the number of centroids(X-Axis). And it's
imeadiatly clear just by a glance that both OpenMP(Orange Line) and MPI
Modes(Green line) greatly outperform the single-core and single-process
Normal Mode(Blue line).

However we can also see that the OpenMP Mode(Orange Line) also
outperforms the MPI Mode(Green line) not only Normal Mode. This is
probably due to all the message exchange between the processes in MPI
Mode that are neccessary for managing the paralellization of the
operations and due to the broadcasting/gathering of the data to/from the
processes.

Finally we remind the reader that the above plot is made by avareging
the behavior of a large number of executions each with a different
initial random centroids placement. And it has beed noticed that if the
inital placement of the centroids is a good one, then the MPI Mode
performs as well as the OpenMP Mode does, so by egaging in a more
sophisticated algorithms for centroids placement we could have comparble
results for the OpenMP and MPI Modes.


    % Add a bibliography block to the postdoc
    
    
    
    \end{document}
